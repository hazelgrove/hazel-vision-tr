% !TEX root = prop.tex
%\clearpage
\section{Aim 1: A Static Semantics for Incomplete Programs}\label{sec:statics}
\vspace{-3px}

Incomplete programs are programs that contain (1) holes, signifying leaves of the syntax tree that have not yet been constructed, (2) type inconsistencies, or (3) binding inconsistencies
(e.g. unbound variables). Incomplete programs are meaningless from the
perspective of a standard static semantics. Our intuition is that
these problems are \emph{local}, so they should be \emph{localized} and not render the entire program statically meaningless. This, in turn, will allow  
\HazelEnv to support editor services that rely on a formal understanding of the type
and binding structure of every edit state.

\subsection{Preliminary Work}

\begin{figure}[t]
\small
$\arraycolsep=4pt\begin{array}{lllllll}
\mathsf{HTyp:} & \htau & ::= &
  \tarr{\htau}{\htau} ~\vert~
  \tnum ~\vert~
  \tehole
\\
\mathsf{HExp:} & \hexp & ::= &
  x ~\vert~
  \hlam{x}{\hexp} ~\vert~
  \hap{\hexp}{\hexp} ~\vert~
  \hnum{n} ~\vert~
  \hadd{\hexp}{\hexp} ~\vert~
  (\hexp : \htau) ~\vert~
  \hehole ~\vert~
  \hhole{\hexp}
\end{array}$
\caption{Syntax of H-types and H-expressions. Metavariable $x$ ranges over variables and $n$ over numerals.}
\label{fig:hexp-syntax}
\vspace{-5px}
\end{figure}


Partially in preparation for this proposal, we conducted preliminary research to define a static
semantics for a simply typed lambda calculus (with a 
single base type, $\tnum$, for simplicity) extended with holes and type
inconsistencies, but no binding inconsistencies (the results summarized below are detailed in a paper accepted at POPL 2017 \cite{popl-paper}). Figure~\ref{fig:hexp-syntax} defines the syntactic
objects of this calculus -- \emph{H-types}, $\htau$,
are types with holes $\tehole$, and \emph{H-expressions}, $\hexp$, are
expressions with holes $\hehole$, and marked type inconsistencies,
$\hhole{\hexp}$. We call marked type inconsistencies \emph{non-empty holes},
because they mark portions of the syntax tree that remain
incomplete and, as we will see, behave semantically much like empty holes. Types and expressions that contain no holes are \emph{complete
  types} and \emph{complete expressions}, respectively.


\begin{wrapfigure}{r}{0.35\textwidth}
\small
\footnotesize
\begin{subequations}
\fbox{$\hsyn{\hGamma}{\hexp}{\htau}$}~~\text{$\hexp$ synthesizes $\htau$}
\begin{equation}\label{rule:syn-var}
\inferrule{ }{
  \hsyn{\hGamma, x : \htau}{x}{\htau}
}
\end{equation}
\begin{equation}\label{rule:syn-ap}
\inferrule{
  \hsyn{\hGamma}{\hexp_1}{\htau}\\
  \arrmatch{\htau}{\tarr{\htau_2}{\htau'}}\\
  \hana{\hGamma}{\hexp_2}{\htau_2}
}{
  \hsyn{\hGamma}{\hap{\hexp_1}{\hexp_2}}{\htau'}
}
\end{equation}
\begin{equation}\label{rule:syn-num}
\inferrule{ }{
  \hsyn{\hGamma}{\hnum{n}}{\tnum}
}
\end{equation}
\begin{equation}\label{rule:syn-plus}
\inferrule{
  \hana{\hGamma}{\hexp_1}{\tnum}\\
  \hana{\hGamma}{\hexp_2}{\tnum}
}{
  \hsyn{\hGamma}{\hadd{\hexp_1}{\hexp_2}}{\tnum}
}
\end{equation}
\begin{equation}\label{rule:syn-asc}
\inferrule{
  \hana{\hGamma}{\hexp}{\htau}
}{
  \hsyn{\hGamma}{(\hexp : \htau)}{\htau}
}
\end{equation}
\begin{equation}\label{rule:syn-ehole}
\inferrule{ }{
  \hsyn{\hGamma}{\hehole}{\tehole}
}
\end{equation}
\begin{equation}\label{rule:syn-hole}
\inferrule{
  \hsyn{\hGamma}{\hexp}{\htau}
}{
  \hsyn{\hGamma}{\hhole{\hexp}}{\tehole}
}
\end{equation}
\end{subequations}
\fbox{$\hana{\hGamma}{\hexp}{\htau}$}~~\text{$\hexp$ analyzes against $\htau$}
\begin{subequations}
\begin{equation}\label{rule:ana-subsume}
\inferrule{
  \hsyn{\hGamma}{\hexp}{\htau'}\\
  \tcompat{\htau}{\htau'}
}{
  \hana{\hGamma}{\hexp}{\htau}
}
\end{equation}
\begin{equation}\label{rule:syn-lam}
\inferrule{
  \arrmatch{\htau}{\tarr{\htau_1}{\htau_2}}\\
  \hana{\hGamma, x : \htau_1}{\hexp}{\htau_2}
}{
  \hana{\hGamma}{\hlam{x}{\hexp}}{\htau}
}
\end{equation}
\end{subequations}
\vspace{-5px}
\caption{The bidirectional H-typing rules for incomplete expressions.\label{fig:ana-synth}}
\end{wrapfigure}
We define the static semantics of H-types and H-expressions as a \emph{bidirectional type
  system}~\cite{Pierce:2000:LTI:345099.345100,DBLP:conf/icfp/DaviesP00,DBLP:conf/tldi/ChlipalaPH05,bidi-tutorial}
around the two mutually defined judgements in Figure~\ref{fig:ana-synth}. Typing contexts, $\hGamma$, map each variable $x \in
\domof{\hGamma}$ to a hypothesis $x : \htau$ in the usual manner. 
%
Derivations of the type analysis judgement, $\hana{\hGamma}{\hexp}{\htau}$,
establish that $\hexp$ can appear where an expression of type $\htau$ is
expected. Derivations of the type synthesis judgement,
$\hsyn{\hGamma}{\hexp}{\htau}$, infer a type from $\hexp$, which is
necessary in positions where an expected type is not available (e.g. at the
top level.) Algorithmically, the type is an ``input'' of the type analysis
judgement, but an ``output'' of the type synthesis judgement.  Making a
judgemental distinction between these two situations is essential in our
action semantics (Section~\ref{sec:actions}).

Type synthesis is stronger than type analysis in that if an expression is
able to synthesize a type, it can also be analyzed against that type, or
any other \emph{consistent} type, according to the \emph{subsumption rule},
Rule (\ref{rule:ana-subsume}). The \emph{H-type consistency judgement}, $\tcompat{\htau}{\htau'}$, that
appears as a premise in the subsumption rule is a reflexive and symmetric
(but not transitive) relation between H-types defined judgementally in
Figure \ref{fig:type-consistency}. This relation coincides with equality
for complete H-types. Two incomplete H-types are consistent if they differ
only at positions where a hole appears in either type. The type hole is
therefore consistent with every type. This notion of H-type consistency
coincides with the notion of type consistency that Siek and Taha discovered
in their foundational work on gradual type systems, if we interpret the
type hole as the $?$ (i.e. unknown) type \cite{Siek06a}. This discovery is quite encouraging, given that gradual typing is also motivated by a 
desire to make sense of one class of ``intermediate edit states,'' or programs that
have not yet been fully annotated with types.

Rule (\ref{rule:syn-lam}) defines analysis for lambda abstractions,
$\hlam{x}{\hexp}$. There is no type synthesis rule that applies  
to this form, so lambda abstractions can appear only in analytic position,
i.e. where an expected type is known.\footnote{It is possible to also define a
  ``half-annotated'' synthetic lambda form, $\lambda x{:}\tau.e$, but for
  simplicity, we leave it out \cite{DBLP:conf/tldi/ChlipalaPH05}.}  Rule
(\ref{rule:syn-lam}) is not quite the standard rule,
%
% CLG says we don't have room 
% reproduced below:
% \begin{equation*}
% \inferrule{
%   \hana{\hGamma, x : \htau_1}{\hexp}{\htau_2}
% }{
%   \hana{\hGamma}{\hlam{x}{\hexp}}{\tarr{\htau_1}{\htau_2}}
% }
% \end{equation*}
which on its own would leave us unable to analyze
lambda abstractions against the type hole.
%because the type hole is not
%immediately of the form $\tarr{\htau_1}{\htau_2}$. 
Although we could solve this problem by defining a rule 
specifically for that case, we 
% \begin{equation*}
% \inferrule{
%   \hana{\hGamma, x : \tehole}{\hexp}{\tehole}
% }{
%   \hana{\hGamma}{\hlam{x}{\hexp}}{\tehole}
% }
% \end{equation*}
instead produce a single rule through the
simple auxiliary \emph{matched arrow type} judgement,
$\arrmatch{\htau}{\tarr{\htau_1}{\htau_2}}$, defined in Figure
\ref{fig:type-consistency}. This judgement leaves arrow types alone and
assigns the type hole the matched arrow type $\tarr{\tehole}{\tehole}$.  
Rule (\ref{rule:syn-ap}), which governs function
application, also makes use of the matched arrow type judgement to
combine what would otherwise need to be two rules for function application:
one for when $e_1$ synthesizes an arrow type, and another for when $e_1$
synthesizes the type hole. Indeed, Siek and Taha needed two application
rules for the same fundamental reason~\cite{Siek06a}. Later work on gradual
typing introduced this notion of type matching to resolve this redundancy \cite{DBLP:conf/popl/CiminiS16,DBLP:conf/popl/GarciaC15,DBLP:conf/popl/RastogiCH12}.

% Rule (\ref{rule:syn-num}) states that numbers synthesize the $\tnum$
% type. Rule (\ref{rule:syn-plus}) states that $\hexp_1 + \hexp_2$ behaves
% like a function over numbers.
% %
% Rule (\ref{rule:syn-asc}) defines type synthesis of expressions of
% ascription form, $\hexp : \htau$. This allows the user to explicitly state
% a type for the ascribed expression to be analyzed against.

% %
% The rules described so far are sufficient to type complete
% H-expressions. The two remaining rules give H-expressions with holes a
% well-defined static semantics. 


\begin{wrapfigure}{l}{0.5\textwidth}
\small
\footnotesize
\noindent\fbox{$\tcompat{\htau}{\htau'}$}~~\text{$\htau$ and $\htau'$ are consistent}
\begin{subequations}\label{eqns:consistency}
\begin{mathpar}
\inferrule{ }{
  \tcompat{\tehole}{\htau}
}

\inferrule{ }{
  \tcompat{\htau}{\tehole}
}

\inferrule{ }{
  \tcompat{\htau}{\htau}
}

\inferrule{
  \tcompat{\htau_1}{\htau_1'}\\
  \tcompat{\htau_2}{\htau_2'}
}{
  \tcompat{(\tarr{\htau_1}{\htau_2})}{(\tarr{\htau_1'}{\htau_2'})}
}
\end{mathpar}
\end{subequations}
\fbox{$\arrmatch{\htau}{\tarr{\htau_1}{\htau_2}}$}~~\text{$\htau$ has matched arrow type $\tarr{\htau_1}{\htau_2}$}
\begin{subequations}
\begin{mathpar}
\inferrule{ }{
  \arrmatch{\tarr{\htau_1}{\htau_2}}{\tarr{\htau_1}{\htau_2}}
}

\inferrule{ }{
  \arrmatch{\tehole}{\tarr{\tehole}{\tehole}}
}
\end{mathpar}
\end{subequations}
\caption{H-type consistency; matched arrow types.}
\label{fig:type-consistency}
\end{wrapfigure}
Gradual type systems do not consider holes in expression position. Rule (\ref{rule:syn-ehole}) states that the empty expression hole
synthesizes the type hole. Similarly, according to Rule
(\ref{rule:syn-hole}), non-empty holes also synthesize the hole type, as long as the inner H-expression  synthesizes some type. By the subsumption rule and the definition of type consistency, holes can be analyzed against any type.

% CLG this is really informative but just takes too much space.
Given these rules, it is instructive to derive that $\hana{incr : \tarr{\tnum}{\tnum}}{\hhole{incr}}{\tnum}$. 
That is, the function variable $incr$ can appear where an expression of number type is needed, as long as the type inconsistency has been marked. Non-empty holes can be used in this way to internalize the ``squiggly red lines'' that many editors display under type inconsistencies. 

The simple static semantics just described has been mechanized using the Agda proof assistant. It serves as the statics for Hazelnut, a structure editor calculus that we return to in Section~\ref{sec:actions}.

\subsection{Proposed Research}

\noindent\textbf{Binding inconsistencies.} In the simple calculus developed so far,
    all variables must be bound before they are used,
    including those in holes. We will extend this system to support reasoning
    in the presence of binding inconsistencies. This 
    opens a new class of type inconsistencies: an unbound variable
    might be used in several places with different, inconsistent types. Tracking
    the types of unbound variables will therefore require tracking and solving typing constraints.  Dagenais and
    Hendren also studied how to reason statically about programs
    with binding errors using a constraint system, focusing on
    Java programs whose imports are not completely known~\cite{DBLP:conf/oopsla/DagenaisH08}. They neither
    considered programs with holes or other type inconsistencies,
    nor formally mathematically specificied their
    technique. However, they provide a useful starting point for us.

\vspace{0.25ex} 
\noindent\textbf{Practical extensions.} The simple calculus
    is only as expressive as the typed lambda calculus with numbers. We have further investigated,
    as an exercise, an extension with binary sum
    types. We propose to scale up the semantics to handle other modern language
    features, focusing initially on functional language
    constructs (so that \HazelEnv can be used to teach courses that
    are today taught using Standard ML, OCaml or Haskell.) This will include recursive and
    polymorphic functions, recursive types, and labeled product (record) and sum types.
    We also propose to investigate ML-style structural pattern
    matching. All of these will require defining new sorts of holes and static
    inconsistencies, including: (1) non-empty holes at the type level, to handle
    kind inconsistencies; (2) holes in label position; (3) label inconsistency
    errors; and (4) holes and type inconsistencies in patterns. 

\vspace{0.25ex} 
\noindent\textbf{Automation.} Although we plan to
    explore some of these language extensions 
    ``manually,'' extending our existing mechanization, we ultimately plan 
    to \emph{automatically}
    generate a statics for incomplete terms from a standard statics for complete terms,
    annotated perhaps with additional information. There is some precedent for
    this in recent work on the Gradualizer, which is capable of
    producing a gradual type system from a standard type system with lightweight
    annotations that communicate the intended polarities of certain
    constructs~\cite{DBLP:conf/popl/CiminiS16}. However, although it provides a good starting point, gradual type systems 
    only consider the problem of holes in 
    types, and the Gradualizer does not support bidirectionally typed languages.
    Our plan is to build 
    upon existing proof automation techniques, e.g. Agda's reflection \cite{van2012engineering}.

