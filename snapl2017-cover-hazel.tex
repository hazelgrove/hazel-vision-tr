% \documentclass{article}
\documentclass[letterpaper,USenglish]{lipics-v2016}
%This is a template for producing LIPIcs articles. 
%See lipics-manual.pdf for further information.
%for A4 paper format use option "a4paper", for US-letter use option "letterpaper"
%for british hyphenation rules use option "UKenglish", for american hyphenation rules use option "USenglish"
% for section-numbered lemmas etc., use "numberwithinsect"
\def\OPTIONConf{0}%
\def\OPTIONArxiv{0}%
 
\usepackage{microtype}%if unwanted, comment out or use option "draft"

%\graphicspath{{./graphics/}}%helpful if your graphic files are in another directory

\bibliographystyle{plainurl}% the recommended bibstyle

% Author macros::begin %%%%%%%%%%%%%%%%%%%%%%%%%%%%%%%%%%%%%%%%%%%%%%%%
\title{Toward Semantic Foundations for Program Editors
%\footnote{This work was partially supported by someone.}
}
\titlerunning{Toward Semantic Foundations for Program Editors} %optional, in case that the title is too long; the running title should fit into the top page column

%% Please provide for each author the \author and \affil macro, even when authors have the same affiliation, i.e. for each author there needs to be the  \author and \affil macros
\author[1]{Cyrus Omar}
\author[1]{Ian Voysey}
\author[2]{Michael Hilton}
\author[1]{Joshua Sunshine}
\author[1]{Claire Le Goues}
\author[1]{Jonathan Aldrich}
\author[3]{Matthew A. Hammer}
\affil[1]{Carnegie Mellon University, Pittsburgh, PA, USA\\
  \texttt{\{comar,iev,sunshine,clegoues,aldrich\}@cs.cmu.edu}}
\affil[2]{Oregon State University, Corvallis, OR, USA\\
\texttt{hiltonm@eecs.oregonstate.edu}}
\affil[3]{University of Colorado Boulder, Boulder, CO, USA\\
  \texttt{matthew.hammer@colorado.edu}}

% \affil[2]{Carnegie Mellon University, Pittsburgh, PA, USA\\
%   \texttt{iev@cs.cmu.edu}}
% \affil[3]{Oregon State University, Corvallis, OR, USA\\
%   \texttt{hiltonm@eecs.oregonstate.edu}}
% \affil[4]{Carnegie Mellon University, Pittsburgh, PA, USA\\
%   \texttt{sunshine@cs.cmu.edu}}
% \affil[5]{Carnegie Mellon University, Pittsburgh, PA, USA\\
%   \texttt{clegoues@cs.cmu.edu}}
% \affil[6]{Carnegie Mellon University, Pittsburgh, PA, USA\\
%   \texttt{aldrich@cs.cmu.edu}}
% \affil[7]{University of Colorado Boulder, Boulder, CO, USA\\
%   \texttt{matthew.hammer@colorado.edu}}

\authorrunning{C. Omar, I. Voysey, M. Hilton, J. Sunshine, C. Le Goues, J. Aldrich, and M. Hammer} %mandatory. First: Use abbreviated first/middle names. Second (only in severe cases): Use first author plus 'et. al.'

\Copyright{the authors}%mandatory, please use full first names. LIPIcs license is "CC-BY";  http://creativecommons.org/licenses/by/3.0/

% \subjclass{Dummy classification -- please refer to \url{http://www.acm.org/about/class/ccs98-html}}% mandatory: Please choose ACM 1998 classifications from http://www.acm.org/about/class/ccs98-html . E.g., cite as "F.1.1 Models of Computation". 
% \keywords{Dummy keyword -- please provide 1--5 keywords}% mandatory: Please provide 1-5 keywords
% Author macros::end %%%%%%%%%%%%%%%%%%%%%%%%%%%%%%%%%%%%%%%%%%%%%%%%%

%Editor-only macros:: begin (do not touch as author)%%%%%%%%%%%%%%%%%%%%%%%%%%%%%%%%%%
\EventEditors{John Q. Open and Joan R. Access}
\EventNoEds{2}
\EventLongTitle{Summit oN Advances in Programming Languages (SNAPL 2017)}
\EventShortTitle{SNAPL 2017}
\EventAcronym{SNAPL}
\EventYear{2017}
\EventDate{May 7--10, 2017}
\EventLocation{Asilomar, California}
\EventLogo{}
\SeriesVolume{42}
\ArticleNo{23}
% Editor-only macros::end %%%%%%%%%%%%%%%%%%%%%%%%%%%%%%%%%%%%%%%%%%%%%%%

\usepackage{etoolbox}
\usepackage{hyperref}

% \usepackage{srcltx}
% \usepackage{goodcharter}
% \usepackage{euler}

% \usepackage{joshuadunfield}
\usepackage{llproof}
%\usepackage{jdproof}
\usepackage{rulelinks}
\usepackage{listings}
\usepackage{graphicx}
\usepackage{wrapfig}
\usepackage{lmodern}
\usepackage{anyfontsize}
\usepackage{stmaryrd}
\SetSymbolFont{stmry}{bold}{U}{stmry}{m}{n}
\usepackage{amsmath,amsthm,amssymb}
\usepackage{thmtools,thm-restate}
\usepackage{todonotes}
\usepackage{enumerate}

% \usepackage[authoryear]{natbib}
% \bibpunct{(}{)}{;}{a}{}{,}

%\mprset{sep=1em}
\def\MathparLineskip{\lineskiplimit=0.9em\lineskip=0.8em plus 0.2em}

\declaretheoremstyle[
  bodyfont=\sl
]{mytheoremstyle}

\lstset{tabsize=2, 
basicstyle=\ttfamily, 
% keywordstyle=\sffamily,
commentstyle=\itshape\ttfamily\color{gray}, 
stringstyle=\ttfamily\color{red},
mathescape=false,escapechar=\#,
numbers=left, numberstyle=\scriptsize\color{gray}\ttfamily, language=ML, moredelim=[il][\sffamily]{?},showspaces=false,showstringspaces=false,xleftmargin=15pt, 
classoffset=0,belowskip=\smallskipamount, aboveskip=\smallskipamount
}
\lstloadlanguages{Java,VBScript,XML,HTML,ML}
\let\li\lstinline

\newcommand{\Hazel}[0]{\textsf{Hazel}}
\newcommand{\HazelEnv}[0]{\Hazel}

\newcommand{\RuleHead}[1]{\text{\raisebox{1em}[0pt]{\ensuremath{\mathsz{\ifnum\OPTIONConf=1 14pt\else 18pt \fi}{#1}}}}~~~~~}

\newcommand{\abort}{\keyword{abort}\xspace}
\newcommand{\xerrs}{\keyword{error}}
\newcommand{\errs}{\;\xerrs}
% \newcommand{\xmatchfailure}{\keyword{match-failure}}
% \newcommand{\matchfailure}{\;\xmatchfailure}
\newcommand{\inj}[1]{\keyword{inj}_{#1}\,}
\newcommand{\Inj}[1]{\inj{#1}}

\newcommand{\rulename}[1]{\text{\normalfont\textsf{#1}}}

\newcommand{\srctyperulename}[1]{\rulename{\textcolor{dDkRed}{S#1}}}
\newcommand{\srcintrorulename}[1]{\srctyperulename{{#1}Intro}}
\newcommand{\srcelimrulename}[1]{\srctyperulename{{#1}Elim}}

\newrulecommand{TAbort}{\targettyperulename{Abort}}

% \newcommand{\subrulename}[1]{\rulename{$\subtype${#1}}}
% \newrulecommand{SubInjDynsum}{\subrulename{Inj$+?$}}
% \newrulecommand{SubDynsumSum}{\subrulename{${+?}{+}$}}
% \newrulecommand{SubRefl}{\subrulename{Refl}}
% \newrulecommand{SubTrans}{\subrulename{Trans}}
% \newrulecommandONE{SubDynsumInj}{\subrulename{${+?}\Inj{#1}$}}

\newcommand{\tytrans}[1]{{|}{#1}{|}}
\newcommand{\ctxtrans}[1]{\tytrans{#1}}

\newcommand{\Rule}[2]{\textsf{#2}}

% \newtheorem{theorem}{Theorem}
\definecolor{Green}{rgb}{0.0, 0.99, 0.0}
\definecolor{light-gray}{rgb}{0.95, 0.95, 0.95}

\begin{document}
\maketitle


\section{Justification statement}


%% SPEC: http://snapl.org/2017/cfp.html
%%
%%
%% The justification statement should briefly explain why the
%% submission is appropriate for SNAPL and summarize the new/different
%% paradigm, perspective, or position.
%%
%% SNAPL welcomes contributions about visionary ideas requiring years
%% of exploration and evaluation, progress on an ongoing, long term
%% research program, lessons from a completed project, including
%% design mistakes, well-argued challenges to accepted ideas and
%% methods, an unexpected connection between two areas of programming
%% languages or a new line of research that builds off of other areas
%% of Computer Science or other disciplines. This list is not intended
%% to be exclusive.


Programming language definitions assign formal meaning to
\emph{complete} programs.
%
Programmers, however, spend a substantial amount of time interacting
with \emph{incomplete} programs -- programs with holes, type
inconsistencies and binding inconsistencies.
%
Unlike complete programs, these incomplete programs are typically not
assigned a formal meaning by a language's semantics.
%
Consequently, the tools that programmers use to \emph{edit} their
(incomplete) programs lack a foundational semantics.

%
%% Semanticists have done comparatively little to formally characterize 1) the static and dynamic semantics of incomplete programs; 2) the 
%% actions available to programmers as they edit and inspect incomplete programs; and 3) the behavior of editor services that suggest likely edit actions to the programmer based on semantic information extracted from the incomplete program being edited, and from programs that the system has encountered in the past.% As such, each tool designer has largely been left to develop their own \emph{ad hoc} heuristics. 

This paper and talk will serve as a vision statement for a research
program that seeks to develop these ``missing'' semantic foundations.
%
%Unlike most existing research on interactive tools, however, our
%contributions are primarily formal in nature.
%
We will outline our planned contributions, which will take the form of
a series of simple formal calculi equipped with a clear mechanized
metatheory.
%
At the same time, we have and will continue to implement these
formalisms as interactive prototypes, culminating in an interactive
programming environment that we refer to as \HazelEnv.
%
Perhaps controversially, we plan to co-design the \HazelEnv~language
with the editor so that we can explore novel interactions of these
designs.
%
Indeed, our exisitng and proposed research methology consists of
mixing two typically desperate activities, both starting from a clean
sheet (but with ML-like languages in mind, i.e., those with types and
higher-order functions):
%
%\begin{itemize}
%\item 
(1) Evolve a theory of typing and operational semantics for
  \emph{incomplete} programs.
%\item 
(2) Evolve an \textbf{editor} (as a formal semantics) for interacting
  with these incomplete programs.
%\end{itemize}

We aim to produce an experience that can emulate (but someday perhaps
surpass) today's user experiences of live lab notebooks (programming
environments designed for data science tasks).
%
However, in reaching this level of experience, we also seek a \emph{compositional} theory of combined editor-and-language semantics.
%
By compositional, we mean that we can ``project out'' a language
feature (e.g., ``functions'', ``sums'', ``products'', ``library for
numbers'', ``library for matrices'', etc.), and that this projection
should give us both the language fragment’s editing behavior and its
typing and operational semantics (for incomplete programs).  In sum,
we are seeking to create a compositional theory of combined
editor-and-language semantics.

To reach this goal, we must lay concrete plans to evolve both our
theory of incomplete programs, as well as our theory for editing them,
both of which are currently rudimentary.
%
Further, we have yet to give an \emph{operational} account of
incomplete programs, focusing so far on their static semantics.
%
Our talk will motivate these directions and outline these frontiers.
%
We are eager to receive feedback from other SNAPL attendees.

\section{Attendance statement}

Cyrus Omar and Matthew Hammer commit to attending SNAPL 2017.
%
Cyrus will present this work.

%% SPEC: http://snapl.org/2017/cfp.html
%%
%% The attendance statement must specify which author(s) commit to
%% attend upon acceptance/invitation, and who will present the
%% work. SNAPL requests a commitment for designated presenters because
%% a good SNAPL contribution is more of a keynote address than a
%% regular paper, which means that the presenters need to be able to
%% guide the discussion during and after the presentation. We expect
%% the author teams to select their best speaker, who is not
%% necessarily the most senior author. The cover page is part of the
%% package given to the reviewers, hence SNAPL submissions are not
%% double-blind.

\end{document}

