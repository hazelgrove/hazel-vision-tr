% !TEX root = prop.tex
\section{Aim 3: An Edit Action Semantics}\label{sec:actions}

\HazelEnv is a structure editor, i.e. a 
program editor where every edit state maps onto a program structure. By
avoiding the parsing step, the problem of reasoning about malformed text is
eliminated.  Programmers interact with structure editors by performing
\emph{edit actions}.

Structure editors have a long history. For example, the Cornell Program
SSynthesizer was developed in the early 1980s \cite{teitelbaum_cornell_1981}. There is still significant   
 interest in structure editors today. For example,  Scratch is a 
structure editor that has achieved success as a tool for teaching children
how to program \cite{Resnick:2009:SP:1592761.1592779}. \texttt{mbeddr} is an editor for a C-like
language \cite{voelter_mbeddr:_2012}. TouchDevelop is an editor for an
object-oriented language \cite{tillmann_touchdevelop:_2011} and Lamdu is an
editor for a functional language similar to Haskell \cite{lamdu}. Most work on structure editors has focused on the user
interfaces that they present. Indeed, this is important work -- presenting a
fluid user interface involving higher-level edit actions is a non-trivial
problem, and some aspects of this problem remain open.

By contrast, the \emph{semantics of edit actions} has received comparatively little
attention. In particular, we seek a formal system that describes how each action acts on the edit
state. 
% %
% \begin{wrapfigure}{R}{0.35\textwidth}
% \includegraphics[width=0.35\textwidth]{impl-overview}
% %\caption{}
% \vspace{-1.3em}
% \end{wrapfigure}
Formalizing the semantics of edit actions is important in \HazelEnv
because it allows us to formally establish a critical and non-trivial invariant,
which we call \emph{sensibility}: that every edit state can be assigned static
and dynamic meaning according to the static and dynamic semantics for incomplete
programs that we have proposed above. 

% The
% edit state of a cell in \HazelEnv is drawn $\zexp_i \Rightarrow
% \htau_i$. It consists of an H-expression with a superimposed
% cursor, which we collectively call a Z-expression, $\zexp_i$ (so named because
% our encoding uses the \emph{zipper} pattern~\cite{huet}, discussed
% below.) Each Z-expression is semantically meaningful, i.e. the underlying
% H-expression synthesizes an H-type, $\htau_i$, according to the static semantics (Section~\ref{sec:statics}).
% This, in turn, implies that it is also dynamically
% meaningful according to the dynamic semantics (Section~\ref{sec:dynamics}).

In addition to maintaining the sensibility invariant,
developing a formal action semantics also allows us to establish that the edit
actions are sufficiently \emph{expressive}. For example, we can mathematically
prove that there is some way to construct every meaningful program (i.e. that we
have not forgotten to define critical term construction actions).

Finally, because we are proposing a \emph{clean-slate} design for \HazelEnv
where we design the static semantics together with the action semantics, we can
explore novel features that blur the boundary between the language
and the editor. We summarize these features below.

\subsection{Preliminary Work}
\newcommand{\cvert}{{\,{\vert}\,}}
\begin{figure}
\small 
\hspace{-3px}$\arraycolsep=2pt\begin{array}{lllllll}
\mathsf{ZTyp} & \ztau & ::= &
  \zwsel{\htau} \cvert
  \tarr{\ztau}{\htau} \cvert
  \tarr{\htau}{\ztau} 
\qquad
\mathsf{ZExp} & \zexp & ::= &
  \zwsel{\hexp} \cvert
  \zexp : \htau \cvert
  \hexp : \ztau \cvert
  \hlam{x}{\zexp} \cvert
  \hap{\zexp}{\hexp} \cvert
  \hap{\hexp}{\zexp} \cvert
  \hadd{\zexp}{\hexp} \cvert
  \hadd{\hexp}{\zexp} \cvert
  \hhole{\zexp}
\end{array}$
\caption{Syntax of Z-types and Z-expressions.}
\label{fig:zexp-syntax}
\vspace{-2ex}
\end{figure}

In preparation for this proposal, we have developed Hazelnut, an action semantics
for the minimal language of H-types and H-expressions described in
Section~\ref{sec:statics}. Portions of this work is scheduled to appear in our forthcoming POPL paper~\cite{popl-paper}. Figure \ref{fig:zexp-syntax} defines the syntax
of Z-types, $\ztau$, and Z-expressions, $\zexp$. A Z-type (resp. Z-expression)
represents an H-type (resp. H-expression) with a single superimposed
\emph{cursor}, indicated as $\zwsel{\htau}$ (resp. $\zwsel{\hexp}$.) The grammar
can be understood as an instance of Huet's well-known \emph{zipper pattern}
\cite{JFP::Huet1997} (which, coincidentally, Huet encountered while implementing
a structure editor).

% The only base cases in these inductive grammars are $\zwsel{\htau}$ and
% $\zwsel{\hexp}$, which identify the H-type or H-expression that the cursor
% is on. All of the other forms correspond to the recursive forms in the
% syntax of H-types and H-expressions, and contain exactly one ``hatted''
% subterm that identifies the subtree where the cursor will be found. Any
% other sub-term is ``dotted'', i.e. it is either an H-type or
% H-expression. Taken together, every Z-type and Z-expression contains
% exactly one selected H-type or H-expression by construction. 


\begin{wrapfigure}{l}{0.55\textwidth}
\small 
\hspace{-3px}$\arraycolsep=3pt\begin{array}{llcllll}
\mathsf{Action} & \alpha & ::= &
  \aMove{\delta} ~\vert~
  \aConstruct{\psi} ~\vert~
  \aDel ~\vert~
  \aFinish\\
\mathsf{Dir} & \delta & ::= &
  \dChildnm{n} ~\vert~
  \dParent\\
\mathsf{Shape} & \psi & ::= &
  \farr ~\vert~
  \fnum \\
& & \vert &
  \fasc ~\vert~
  \fvar{x} ~\vert~
  \flam{x} ~\vert~
  \fap ~\vert~
  % \farg ~\vert~
  \fnumlit{n} ~\vert~
  \fplus\\
& & \vert &
  {\color{gray}\fnehole}
\end{array}$
\caption{Syntax of actions.}
\label{fig:action-syntax}
\end{wrapfigure}
The heart of Hazelnut's editing model is its \emph{bidirectional action
  semantics}.  Figure~\ref{fig:action-syntax} defines the syntax of
\emph{actions}, $\alpha$, some of which involve \emph{directions},
$\delta$, and \emph{shapes}, $\psi$. Expression actions are governed by two mutually defined judgements, the
\emph{synthetic action judgement}:
$\performSyn{\hGamma}{\zexp}{\htau}{\alpha}{\zexp'}{\htau'}$
and \emph{the analytic action judgement}, 
$\performAna{\hGamma}{\zexp}{\htau}{\alpha}{\zexp'}$. 
In some Z-expressions, the cursor is in a type ascription, so we also need
a \emph{type action judgement}, $\performTyp{\ztau}{\alpha}{\ztau'}$, pronounced ``performing $\alpha$ on $\ztau$
results in $\ztau'$''.

\begin{wrapfigure}{r}{0.5\textwidth}
\small
  \label{ex2}
\begin{center}
\colorbox{light-gray}{\hspace{44px}  assume $incr : \tarr{\tnum}{\tnum}$ \hspace{44px}}
$\arraycolsep=4px
\begin{array}{|r||l|l||l|}
\hline
\# & \textbf{Z-Expression} &
\textbf{Next Action} 
\\
\hline
1 &
\zwsel{\hhole{}} &
\aConstruct{\fvar{incr}} \hphantom{~\,\,~~}
\\ 2 &
\zwsel{incr} &
\aConstruct{\fap}
\\ 3 &
incr(\zwsel{\hhole{}}) &
\aConstruct{\fvar{incr}}
\\ 4 &
incr(\hhole{\zwsel{incr}}) &
\aConstruct{\fap}
\\ 5 &
incr(\hhole{incr(\zwsel{\hhole{}})}) \hphantom{~~~~} &
\aConstruct{\fnumlit{3}}
\\ 6 &
incr(\hhole{incr(\zwsel{\hnum{3}})}) &
\aMove{\dParent}
\\ 7 &
incr(\hhole{\zwsel{incr(\hnum{3})}}) &
\aMove{\dParent}
\\ 8 &
incr(\zwsel{\hhole{incr(\hnum{3})}})&
\aFinish
\\ 9 &
incr(\zwsel{incr(\hnum{3})}) &
\textrm{---}
\\ \hline
\end{array}
$\end{center}\vspace{-10px}
\caption{Applying the increment function.}
\label{fig:second-example}
\end{wrapfigure}
In the interest of brevity, we will not reproduce all of the rules, but will
instead give the intuition in the form of the example edit
sequence shown in Figure~\ref{fig:second-example}. Every edit state here, after \emph{cursor erasure}, synthesizes a type according
to the semantics developed in Sec. \ref{sec:statics}. 
We begin on Line 1 by constructing the variable $incr$. Line 2 then
constructs the application form with $incr$ in function position, leaving
the cursor on a hole in the argument position. Notice that we did not need
to construct the outer application form before identifying the function
being applied. 

%
We now wish to apply $incr$ again, so we perform the same action on Line 3
as we did on Line 1, i.e. $\aConstruct{\fvar{incr}}$. Na\"ively, performing such an action would result in an 
edit state that is ill-typed after \emph{cursor erasure}:
$incr(\zwsel{incr})$. 
The argument of
$incr$ must be of type $\tnum$ but here it is of type
$\tarr{\tnum}{\tnum}$. The action semantics avoids this problem by inserting a non-empty hole, marking the type inconsistency as described in Sec. \ref{sec:statics}. Actions are \emph{type aware}, so a non-empty hole is inserted only if necessary.

The programmer could alternatively have performed the $\aConstruct{\fap}$
action on Line 15, which would result in the following well-typed edit state (according to our static semantics Section~\ref{sec:statics}):
$incr(\hhole{}(\zwsel{\hhole{}}))$. 
The problem is that the programmer is not able to identify the intended
function before constructing the function application form. We aim to avoid this prescriptive ``outside-in'' editing style.
% This stands in
% contrast to Lines 14-15.

Lines 4--5 proceed to apply the inner
mention of $incr$ to a number literal, $\hnum{3}$. Finally, Lines
6--7 move the cursor to the non-empty hole. Line 8 performs a $\aFinish$
action, which removes the hole if the type of the
expression inside it is consistent with the expected type, as it now
is. This results in the desired final edit state on Line 9. In
practice, the editor would automatically perform the $\aFinish$ action as
soon as it becomes possible, but for simplicity, Hazelnut currently formally requires
that it be performed explicitly.

The key 
metatheorem, \emph{sensibility}, is formally stated as follows:% , so the
% reader is encouraged to think about the relevant cases of these proofs as
% we present the rules.
\begin{theorem}[Action Sensibility]
  \label{thrm:actsafe} Both of the following hold:
  \begin{enumerate}[itemsep=0px,partopsep=0px,topsep=0px]
  \item If $\hsyn{\hGamma}{\removeSel{\zexp}}{\htau}$ and
    $\performSyn{\hGamma}{\zexp}{\htau}{\alpha}{\zexp'}{\htau'}$ then
    $\hsyn{\hGamma}{\removeSel{\zexp'}}{\htau'}$.
  \item If $\hana{\hGamma}{\removeSel{\zexp}}{\htau}$ and
    $\performAna{\hGamma}{\zexp}{\htau}{\alpha}{\zexp'}$ then
    $\hana{\hGamma}{\removeSel{\zexp'}}{\htau}$.
  \end{enumerate}
\end{theorem}
\noindent In other words, if an edit state (i.e. a Z-expression) is
statically meaningful, i.e. its cursor erasure, $\removeSel{\zexp}$, is well-typed, then
performing an action on it leaves the resulting edit state statically
meaningful. In particular, the first clause of Theorem \ref{thrm:actsafe}
establishes that when an action is performed on an edit state whose cursor
erasure synthesizes an H-type, the result is an edit state whose cursor
erasure also synthesizes some (possibly different) H-type. The second
clause establishes that when an action is performed using the analytic
action judgement on an edit state whose cursor erasure analyzes against
some H-type, the result is a Z-expression whose cursor erasure also
analyzes against the same H-type.

This and a number of other theorems that establish that the rules governing
the various actions are satisfying (e.g. that movement indeed allows one 
to move the cursor to anywhere, and that construction indeed allows one 
to construct any meaningful expression), have been mechanized in Agda.




\subsection{Proposed Work}
The static semantics and action semantics developed in our preliminary work on Hazelnut lay the foundation for a number of proposed problems:

\vspace{0.25ex}
\noindent\textbf{Practical extensions.} In Section~\ref{sec:statics}, we
discussed our plans to add a number of modern language features to the static
semantics. These require corresponding extensions to the action semantics, and
we plan to maintain the static and action semantics in lockstep. 

\vspace{0.25ex}
\noindent\textbf{Automation.} Similarly, in Section~\ref{sec:actions}, we
discussed the need to automatically generate certain ``boilerplate''
constructs. This is perhaps even more important in the action semantics. For
example, the zipper structure and the movement rules could easily be generated
automatically by following the term structure. Ultimately, we would like to
produce an \emph{Editorializer}, i.e. a transformation that produces a sensible
and expressive action semantics from a lightly annotated definition of the
static semantics of complete expressions. There is precedent in various
structure editor generators, e.g. Barista~\cite{ko_barista:_2006}. Our
contribution would be in producing a structure editor generator with a
correctness guarantee. 

\vspace{0.25ex}
\noindent\textbf{Programmable Actions.} Our language of actions is intentionally primitive. However, even now it
acts much like a simple imperative command language. We propose to develop the
notion of an \emph{action macro}, whereby 
functional programs could themselves be lifted to the level of actions to
compute non-trivial compound actions. Such compound actions would give a
uniform description of transformations ranging from the simple---like
``move the cursor to the next hole to the right''---to quite complex whole
program refactorings, while remaining subject to the core semantics. Using proof automation, it should be possible to prove that an action macro implements derived action logic that 
is admissible with respect to the core semantics. This would eliminate the possibility of ``edit-time'' errors.

\vspace{0.25ex}
\noindent\textbf{Type-Specific Projections.} Another important research
direction lies in exploring how types can be used to control  
the presentation of expressions in the editor. Following an
approach developed by PI Aldrich in a textual setting of \emph{type-specific
languages} (TSLs), it should be possible to have the type that an
expression is being analyzed against define a \emph{type-specific projection} defined by elaboration to the underlying static semantics and action semantics of \HazelEnv \cite{TSLs}. For example, the matrix projection shown in Figure \ref{fig:hazel-mockup} need not be built in to \HazelEnv. Instead, the \li{Numerics} library provider could introduce this logic. This line of research is also closely related to \emph{active code completion}, which investigated type-specific code generation interfaces \cite{Omar:2012:ACC:2337223.2337324}. The first author on that work is advised by PI Aldrich and contributed to the writing of this proposal.

\vspace{0.25ex}
\noindent\textbf{Interactive Semantic Documentation.} Finally, we propose
to develop a semantics of semantic comments, 
i.e. comments that mention semantic structures. These would be subject to
the same operations, e.g. renaming, as other structures, helping to address
the problem of comments becoming inconsistent with code. Going further, it should be possible to use the 
action semantics as the foundation for a ``tutorial'' system that introduces portions of a program in non-linear
sequence. The learner can proceed through the tutorial with the full array of
editor services available at each intermediate edit state. There is evidence to
suggest that revealing pieces of programs in sequence and interacting with
intermediate edit states can be valuable in educational settings~\cite{Bennedsen:2005:RPP:1047344.1047413,Paxton:2002:LPL:771322.771332}.  
