% !TEX root = prop.tex

%%% Summary
\clearpage
\rfoot{}

\paragraph{Overview.}
Programming language definitions assign meaning to \emph{complete} programs. 
Programmers, however, spend a substantial amount of time interacting with \emph{incomplete} programs using tools like program editors and live programming environments (which interleave editing and evaluation.)  
Semanticists have paid comparatively little attention to these interactions, so the designers of tools like these lack foundational semantic principles comparable to those available to language designers. % Instead, tool designers are often left to develop  heuristic approaches and evaluate these approaches using only qualitative and empirical methods.
The objective of the proposed research is to develop the ``missing'' semantic principles for interactive programming tools. We propose:

\begin{enumerate}[noitemsep,nolistsep,leftmargin=*,align=left]
\item a \textbf{static semantics for incomplete programs} that assigns static meaning to programs with \emph{holes}, \emph{type inconsistencies}, \emph{binding inconsistencies} and other local, transient problems; 
\item a \textbf{dynamic semantics for incomplete programs} that assigns dynamic meaning to incomplete programs and supports ``edit-and-resume'' functionality, thereby tightening the live programming feedback loop;  
\item an \textbf{action semantics} that captures the process of editing a program using structured edit actions and maintains a powerful semantic invariant: that every intermediate edit state can be assigned static and dynamic meaning according to the aforementioned semantics for incomplete programs; and 
\item a \textbf{statistical action suggestion semantics}, which serves as a foundation for advanced editor features, like semantic code completion and automatic program repair, that need to generate both \emph{semantically valid} and \emph{statistically likely} code snippets and actions.
\end{enumerate}

To ensure that these individual developments lead toward a coherent and practical \emph{theory of interactive programming}, we plan to integrate them into a \emph{live lab notebook} programming environment, \HazelEnv. By embracing a clean-slate integrative approach, we can investigate the semantics of novel constructs that are defined within programs but control the programming environment, notably \textbf{programmable edit action macros}, \textbf{type-specific projection macros} and \textbf{semantic, interactive documentation}.

Although our proposed contributions are primarily {mathematical}, we will also conduct small \textbf{pilot studies} involving in order to 1) gather data for the action suggestion system; 2) iterate on \HazelEnv's design and evaluate whether our ``semantics-first'' tool design methodology can scale to produce a tool that allows programmers to productively engage in non-trivial (if not yet large-scale) programming tasks; and 3) evaluate whether the interactive documentation system we have proposed improves tutorial comprehension, relative to an approach that relies on non-interactive documentation.
%; and 3) extract broader insights about the learning and programming process from the uniquely rich datasets that Hazel will generate.

\vspace{-2px}\paragraph{Intellectual Merit.}

% This proposal describes a general-purpose framework that assigns
% precise meaning to the acts of programming (viz., editing, debugging
% and testing), as well as the mixture and interposition of these acts.
% The proposed framework is closely related to historical work on
% structure editors, which enrich the editing environment with an
% editing language of structural, invariant-preserving edits.  This
% framework goes further than this past work, however, since it focuses
% on invariants from the perspective of a foundational type system (a la
% core ML), and its static and dynamic semantics both support partial
% programs whose syntax and type-based invariants may be incomplete.

The proposed research aims to lay principled semantic foundations under 
tools that are in wide use and under active development, but today rely on variously \emph{ad hoc} heuristics (e.g. semantics-aware
editors, live programming interfaces and notebook interfaces.) The initial contributions will 
take the form of several calculi in the style of the typed lambda calculus, each based in part on 
well-understood logical systems (e.g. contextual model logic) and equipped with 
rich mechanized metatheory. 
These individual calculi will be combined and extended in the design and development of \HazelEnv. The pilot studies that we conduct will generate qualitative and quantitative data that we plan to both publish and utilize internally as we iterate on Hazel's design. Taken together, these contributions have the potential to open up novel research directions for semanticists, tool designers and statisticians both individually and in interdisciplinary collaboration. 
%
% Finally, by recording the interaction in this proposed implementation,
% the proposed system generates corpuses of fine-grained program
% evolution such that every intermediate state is well-formed, and thus
% readily analyzable by statistical models that leverage the invariants
% and types of the evolving program.  Based on these corpuses, the
% programming environment gives data-driven suggestions to help complete
% partial programs.

\vspace{-2px}\paragraph{Broader Impacts.}
Achieving the vision outlined above will change how interactive programming tools are designed, much as achievements in programming language theory have had a broad impact on language design. This will ultimately make the job of the programmer easier because future tools informed by \HazelEnv will 1) eliminate malformed edit states and edit states where the editor cannot provide assistance; 2) improve the quality of the assistance offered by combining powerful statistical and semantic techniques; and 3) tighten the ``feedback loop'' between editing and evaluation. Previous efforts toward similar ends have historically been particularly helpful to novice programmers, and we also aim to specifically target this group with \HazelEnv by incorporating tutorials developed using the interactive documentation system proposed above into the undergraduate curriculum at CMU and CU Boulder, publicizing these tutorials online (\HazelEnv has an open source browser-based implementation) and introducing these tutorials to the Girls of Steel, the largest all-girls FRC robotics team in the country, which PI Le Goues actively mentors.
